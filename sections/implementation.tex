\section{Implementación}

Se eligió Haskell porque, como lenguaje funcional, facilita la implementación de algoritmos de listas como MTF e IMTF al ofrecer una programación inmutable, una sintaxis concisa y un sistema de tipos robusto. Esto permite trabajar con listas de forma clara, eficiente y segura.\\

La implementación puede ser hallada en lo siguiente enlaces:

\begin{itemize}
    \item MTF: \url{https://github.com/chamale-rac/mtf/blob/main/implementation/mtf/mtf.hs}
    \item IMTF: \url{https://github.com/chamale-rac/mtf/blob/main/implementation/imtf/imtf.hs}
\end{itemize}

A continuación se presenta el enlace al repositorio, en el cual bajo el subdirectorio \textit{implementation/*} se adicionalmente los archivos compilados (\textit{.exe}) listos para ser ejecutados en un OS Windows. \\

Github: \url{https://github.com/chamale-rac/mtf}