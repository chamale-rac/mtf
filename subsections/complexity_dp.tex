\subsection{\textit{\textbf{Algoritmo DP}}}
En este caso, al utilizar un enfoque \textit{bottom-up}, se ejecuta de manera constructiva, generando cada caso posible en lugar de utilizar recursión o ramificación. Esto proporciona un tiempo de complejidad estable y dependiente del tamaño de los casos necesarios.\\

Debido a esto, el análisis de la complejidad temporal se vuelve sencillo. Al no haber recursión real, sino solo llamadas a la estructura de datos, la operación relevante en nuestro algoritmo son los ciclos anidados. Dado que tenemos dos ciclos, uno que recorre \(n\)  y otro que recorre \(m\), el tiempo se puede definir como \(O(m \times n) \).