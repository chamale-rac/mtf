\section*{Problema 1}
Tomamos R\((n)\) como el valor óptimo para la cuerda de longitud \(n\). Suponiendo que existe una solución óptima para la cuerda completa construida por soluciones óptimas de los trozos de la cuerda, se puede representar R\((n)\) de la siguiente manera:

\[ R(n) = R(1) + R(2) + R(3) + \dots + R(N-1) \]

Pero, al haber un límite de piezas, disminuyen las opciones para cortar la cuerda, por lo que asumir lo anterior sería incorrecto. Lo anterior dado que, la solución óptima ya no podría construirse a partir de soluciones óptimas de subproblemas más pequeños, significando que el rod-cutting problem con límite de piezas no posee unas subestructura óptima.\\

Supongamos una situación tal que\dots
\begin{center}
\begin{tabular}{cccccccccc}
\toprule
n & 0 & 1 & 2 & 3    & 4 & 5 & 6 & 7 & 8 \\ \hline
p & 0 & 1 & 3 & 3.54 & 4 & 5 & 6 & 7 & 8 \\ \hline
\end{tabular}
\end{center}
\hfill \break

Imaginemos que tenemos una cuerda de longitud 6, su solución óptima sería cortarla en 3 trozos de 2, ganando así un total de 9. Ahora, imaginemos que \(l_2 = 1\), nos veríamos obligados a alterar la solución dada la limitante a 2 cortes de 3, ganando 7. ¿El problema?, esta es una solución subóptima con respecto al valor total.\\

Concluimos entonces que este problema no posee una subestructura óptima porque los subproblemas que se obtienen de él no son independientes. La independencia entre subproblemas se refiere a que computar la solución de un subproblema no depende de los recursos de, o no se ve afectada (según las condiciones del problema) por la solución para otro subproblema del mismo problema.