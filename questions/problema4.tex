\section*{Problema 4}
\subsection*{\textit{\textbf{Inciso A}}}

\begin{center}
    \begin{tikzpicture}[node distance={20mm}, thick, main/.style = {draw, circle}] 
    \node[main] (1) {1}; 
    \node[main] (4) [below of=1] {4}; 
    \node[main] (3) [right of=4] {3};
    \node[main] (2) [right of=1] {2}; 
    
    \draw[edge] (1) to node[above] {4} (2);
    \draw[edge] (1) to node[left] {5} (4);
    \draw[edge] (4) to node[below] {3} (3);
    \draw[edge] (3) to node[right] {-10} (2);
    
    \end{tikzpicture}
\end{center}

\begin{align*}
    E & = \text{Arreglo de aristas.} \\
    V & = \text{Arreglo de vértices.} \\
    d & = \text{Arreglo de costos.}
\end{align*}

\subsubsection*{\textbf{Paso 1}}
\begin{center}
\begin{tabular}{@{}cccc@{}}
\toprule
1 & 2 & 3 & 4 \\ \midrule
0 & ∞ & ∞ & ∞ \\ \bottomrule
\end{tabular}
\end{center}

\subsubsection*{\textbf{Paso 2}}
\text{ Dado } d[2] > d[1] + 4, \text{ entonces } d[2] = 4 \text{ y } $\pi$[2] = 1.

\subsubsection*{\textbf{Paso 3}}
\text{ Dado } d[4] > d[1] + 5, \text{ entonces } d[4] = 5 \text{ y } $\pi$[4] = 1.

\subsubsection*{\textbf{Paso 4}}
\text{ Dado } d[3] > d[4] + 3, \text{ entonces } d[3] = 8 \text{ y } $\pi$[3] = 4.

\subsubsection*{\textbf{Paso 5}}
\text{ Dado } d[2] > d[3] + (-10), \text{ entonces } d[2] = -2 \text{ y } $\pi$[2] = 3.

\subsubsection*{\textbf{Paso 6}}
\begin{center}
\begin{tabular}{@{}cccc@{}}
\toprule
1 & 2  & 3 & 4 \\ \midrule
0 & ∞  & ∞ & ∞ \\ 
0 & 4  & ∞ & ∞ \\ 
0 & 4  & ∞ & 5 \\ 
0 & 4  & 8 & 5 \\ 
0 & -2 & 8 & 5 \\ \bottomrule
\end{tabular}
\end{center}
\hfill \break

\text{ El camino más corto está dado por:} $1 \rightarrow 4 \rightarrow 3 \rightarrow 2$.


\subsection*{\textit{\textbf{Inciso B}}}
Este problema exhibe una subestructura óptima al dividirse en subproblemas más pequeños hasta alcanzar el caso base. Se requiere determinar el camino más corto hacia un nodo \(u\), que conecte con el nodo objetivo \(v\). Además, se necesita conocer el camino más corto hacia un nodo que conecte con \(u\), y así sucesivamente, hasta llegar al caso base. Estas divisiones del problema en subproblemas más pequeños generan una subestructura óptima, donde los problemas pueden superponerse debido a la posible necesidad de calcular el camino más corto hacia cierto nodo múltiples veces.

\subsection*{\textit{\textbf{Inciso C}}}
Tomando en cuenta lo anterior, la relación de recurrencia puede expresarse de la siguiente forma:

\[
d[v] = 
\begin{cases}
0, & \text{si } s = v \\
\underset{(u, v) \in E}{min}(d[v], d[u] + w(u, v)), & \text{si } s \neq v
\end{cases}
\]

Esta relación de recurrencia indica que el camino más corto hacia el nodo \(v\) es el mínimo entre el camino más corto conocido hasta el momento hacia \(v\) (representado por ∞ en el primer paso) y la distancia más corta hacia un nodo \(u\) que se conecte con \(v\), más el costo de la transición en sí. Esta condición asegura que la distancia más corta se refine continuamente a medida que se revisan todos los nodos que de alguna manera están conectados con \(v\).